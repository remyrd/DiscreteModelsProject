\documentclass[11pt]{article}   % list options between brackets
\usepackage{listings,color}              % list packages between braces

% type user-defined commands here
\lstset{
    basicstyle=\footnotesize\ttfamily,
    keywordstyle=\color{blue},
    morekeywords={machine, process},
    breakatwhitespace=false,
    breaklines=true,   
}

\begin{document}

\title{T-79.4101 Programming Assignment 1: Process Re-Assignment}   
\author{Remy Rojas 486404}  % Your name and student number
\date{March 1, 2015}            % type date between braces
\maketitle

\section{Problem Description}    

The problem consists of a given number of processes which need to be reassigned in order to find the configuration with the smallest cost. Cost can be of two types: 
\begin{itemize}
    \item Moving a process from machine A to B generates a cost induced only by the process
    \item When a certain capacity is breached in a machine, a cost is induced by the excess resource consumption on the machine. Every machine has its own capacity.
\end{itemize}
Additionally 3 constraints must be satisfied for a configuration to be a valid solution:
\begin{itemize}
    \item MCConst: Machines have a maximum capacity (always greater than the one mentioned previously). 
    If this capacity is breached the machine crashes, so a new process can not be moved to a certain machine if the latter does not have enough capacity.
    \item SSConst: Machines are grouped into Locations and Processes are grouped into Services. The Minimum Spread of a service is a property each service has, which denotes the number of locations in which processes belonging to the service must be. In other words, each service must have representatives in at least X locations.
    \item SCConst: Stipulates two processes from the same service can not run in the same machine.
\end{itemize} 
 
 

\section{Algorithm}       

The local search is performed with steepest descent, taking the process with the least cost
and probing every machine available looking for the best cost difference amongst neighbors.
Neighbors are defined by two assignments with one differing process-machine assignment.
Once all processes were moved to the best option (if available) we implement simulated annealing
by generating random neighbors a vast number of times without checking costs (only constraint satisfaction).
The number of random neighbors to which we move is twice the number of processes.

\begin{lstlisting}
while True:
    
    while iterations < number_processes:
        
        process = find_minimum_cost_process()
        
        for machine in assignment.machines:
            if (constraints_satisfied(process, machine)) and (local_costs_improvement(machine, process)):
                candidates.add(machine)
        
        if candidates not empty:
            best_machine = min(candidates)        
            assignment.update(best_machine, process)
        
        if global_cost_improvement(assignment, original) < current_global_cost:
            dump(assignment)    
        
        iterations ++
    
    randomize(assignment)
    
\end{lstlisting}



\section{Experiments}   

The programming and experimentation were carried out on Cloud9 IDE: a cloud based service running a VM with 512MB RAM memory.

The program succesfully improves the intanced initial solution. Although the upper bound of the solution is not reached within 5min of testing for large instances.
The checker provided in the assignment does not complain about constraint violation when probing solutions,
and the global costs calculated in the program (outputed to the console) correspond to the ones returned by the checker.

\section{Conclusions}         

Unfortunately the results do not reache the upper boundary expected. Still this greatly depends on the
function charged to generate a random assignment. Given enough time it should get to a very good solution.
The settings in the experiment (cloud VM with 512MB RAM) are not optimal and way below standart physical machines.
Nevertheless, the algorithm can be improved by looking into machines with the largest free capacity and prioritizing ignoring
those where soft capacities overflow rather than checking for the smallest move cost.

% Uncomment if you have references
% \begin{thebibliography}{9}
%  % type bibliography here
% \end{thebibliography}

\end{document}
